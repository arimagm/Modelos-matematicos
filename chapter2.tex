\chapter{Álgebra de matrices y vectores aleatorios}

\textcolor{red}{Carlos para la redacción de este capítulo apoyese de la sección 2.5 RANDOM VECTORS AND MATRICES, del Johnson y del capítulo 3 del Jonhson, Sample Geometry and Random Sampling }

Dadas $X_{1},...,X_{p}$, $p$ variables aleatorias, con $n$ datos cada variable, se tiene la siguiente matriz de datos:


$$
\mathbf{X}=\left( 
\begin{array}{cccc}
x_{11} & x_{12} & \cdots  & x_{1p} \\ 
x_{21} & x_{22} & \cdots  & x_{2p} \\ 
\vdots  & \vdots  & \ddots  & \vdots  \\ 
x_{n1} & x_{n2} & \cdots  & x_{np}
\end{array}
\right)
$$
De esta matriz $\mathbf{X}$, se calculas el vector de promedios, la matriz de covarianzas y la matriz de correlaciones.

Vector de promedios
$$
\bar{\mathbf{x}}=(\bar{x}_{1},\bar{x}_{2},...,\bar{x}_{p})
$$


Matriz de covarianzas:

$$
S=\left( 
\begin{array}{cccc}
s_{11} & s_{12} & \cdots  & s_{1p} \\ 
s_{21} & s_{22} & \cdots  & s_{2p} \\ 
\vdots  & \vdots  & \ddots  & \vdots  \\ 
s_{p1} & s_{p1} & \cdots  & s_{pp}
\end{array}
\right)
$$
donde, los elementos de la matriz $S$, están dados por 

$$
s_{ik}=\frac{1}{n}\sum_{j=1}^n(x_{ji}-\bar{x}_{i})(x_{jk}-\bar{x}_{k})
$$
donde $i=1,2,...,p$,  y $k=1,2,3,...,p$.

Por otra parte, la matriz de correlaciones está dada por 

$$
R=\left( 
\begin{array}{cccc}
r_{11} & r_{12} & \cdots  & r_{1p} \\ 
r_{21} & r_{22} & \cdots  & r_{2p} \\ 
\vdots  & \vdots  & \ddots  & \vdots  \\ 
r_{p1} & r_{p1} & \cdots  & r_{pp}
\end{array}
\right)
$$
donde sus elementos están dados por 

$$
r_{ik}=\frac{s_{ik}}{\sqrt{s_{ii}}\sqrt{s_{kk}}}=\frac{\sum_{j=1}^n(x_{ji}-\bar{x}_{i})(x_{jk}-\bar{x}_{k})}{\sqrt{\sum_{j=1}^n(x_{ji}-\bar{x}_{i})^2}\sqrt{\sum_{j=1}^n(x_{jk}-\bar{x}_{k})^2}}
$$


donde $i,k=1,2,...,p$.




Note que si

$$
V^{1/2}=\left( 
\begin{array}{cccc}
\sqrt{s_{11}} & 0             & \cdots  & 0 \\ 
0             & \sqrt{s_{22}} & \cdots  & 0 \\ 
\vdots        & \vdots        & \ddots  & \vdots  \\ 
0             & 0             & \cdots  & \sqrt{s_{pp}}
\end{array}
\right)
$$ entonces existe una relación entre la matriz de correlación y la
matriz de varianzas y covarianzas $$
V^{1/2}RV^{1/2}=S
$$ luego la matriz de correlaciones está dada por $$
R=(V^{1/2})^{-1}S(V^{1/2})^{-1}
$$