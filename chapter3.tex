\chapter{Probabilidad}

Propósito: Entender el concepto de espacio muestral, espacio de eventos, eventos simples y compuestos, función de probabilidad, espacio de probabilidad y  probabilidad de un evento.


%%%%%%%%%%%%%%%%%%%%%%
\section{Antecedentes}

%%%%%%%%%%%%%%%%%%%%%%%%%%%%%%%%%
\section{Teoría de conjuntos}


\begin{definition}[Complemento de un conjunto]
El complemento de un conjunto $A$ con respecto a $\Omega$ denotado por $A^{c}$, es el conjunto de todos los puntos que están en $\Omega$ pero que no están en $A$.
\end{definition}

\begin{definition}[Unión de conjuntos]
Dado $A$ y $B$ dos conjuntos de $\Omega$, la unión de $A$ y $B$, dado por $A\cup B$, es el conjunto que contiene todos los elementos que están $A$ ó $B$ ó de ambos.
\end{definition}

\begin{definition}[Intersección de conjuntos]
Dado $A$ y $B$ dos conjuntos de $\Omega$, la intersección de $A$ y $B$, dado por $A\cap B$, es el conjunto que contiene todos los elementos que están en $A$ y $B$.
\end{definition}

\begin{definition}[Conjuntos disjuntos]
Dos conjuntos $A$ y $B$ definidos en $\Omega $, se dice que son  \textbf{mutuamente excluyentes} o  \textbf{disjuntos} si no tienen elementos en común, es decir, si $A\cap B=\varnothing$, donde $\varnothing$ es el conjunto nulo.
\end{definition}

Así si $\Omega$ es el conjunto universo, $A$ y $B$ subconjuntos de $\Omega$, entonces

\begin{enumerate}[1]
\item $A^{c}$ tiene los elementos de $\Omega$ que no están en $A$.

\item $A\cup B$ son los elementos de $\Omega$ que están en $A$ ó $B$ ó en ambos.
\item $A\cap B$ son los elementos de $\Omega$ que están en $A$ y en $B$.

\end{enumerate}


Otras propiedades importantes, son las siguientes:
\begin{enumerate}[1]

\item $A\subseteq B$ sí $\forall$ $x\in A,x\in B$

\item $A\subset B$ si todo elemento de $A$ está en $B$, además $\exists $ $x\in B$ tal que $x\notin A$ ($A$ es subconjunto propio de $B$).

\item $A-B$ son los elementos de $\Omega$ que están en $A$ pero que no están en $B$; es decir, el conjunto de los elementos $A-B$ son las $x\in A$ pero $\notin B$; entonces $a-B=A\cap B^{c} $ 
\item $\varnothing$ es el conjunto vacío.
\item $\mathcal{P}(A)$ es el conjunto potencia de $A$ y denota todos los subconjuntos de $A.$ Si $A$ es finito con $n$ elementos entonces $\mathcal{P}(A)$ tiene $2^{n}$ elementos.
\end{enumerate}

\begin{exercise}
Sea $U=\{a,$ $b,$ $c,$ $d,$ $e,$ $f\}$, $A=\{a,b,c\}$, $B=\{b,c,d\}$ 
\end{exercise}


\begin{solution}
\begin{itemize}[b]
\item $A^{c} =\{d,e,f\} $
\item $A\cup B =\{a,b,c,d\}$
\item $A\cap B =\{b,c\}$
\item $A-B =\{a\}$
\item $\mathcal{P}(A) =\{\varnothing
,\{a\},\{b\},\{c\}.\{a,b\},\{a,c\},\{b,c\},\{a,b,c\}\}$
\item $\mathcal{P}(B) =\{\varnothing
,\{b\},\{c\},\{d\},\{b,c\},\{b,d\},\{c,d\},\{b,c,d\}\}$
\item $\mathcal{P}(U) =\{\}$
\end{itemize}    
\end{solution}


%%%%%%%%%%%%%%%%%%%%%%%%%%%%%%%%%
\subsection{Practica con r}



\begin{lstlisting}[language=R]

        #--- Ejemplo

        U<-c("a","b","c","d","f","e")
        A<-c("a","b","c")
        B<-c("b","c","d")

        union(A,B)      # unión

        intersect(A,B)  # insección

        setdiff(A,B)    # diferencia

        setdiff(U,A)    # Complemento
\end{lstlisting}


\begin{lstlisting}
       #--- Conjunto potencia

       conjunto<-U

       N <- length(conjunto) 

       re<-list()

       for(i in 1:N)
          {
            re[[i]]<- combinations(N, i, conjunto)  # el orden no importa
          }
\end{lstlisting}

%%%%%%%%%%%%%%%%%%%%%%%%%%%%%%%%%
\subsection{Álgebra de conjuntos}


\begin{property}
\begin{itemize}[b]
    \item Si $A\subset B$ entonces $A\cup B=B$.
    \item Si $A,B\in \Omega$ entonces $A\cap B=B\cap A$.
    \item Si $A\subset B$ entonces $A\cap B=A$.
    \item Dados $A$ y $B$ dos conjuntos de $\Omega$, entonces
    \begin{eqnarray*}
A\cup A^{c}  & = &\Omega \\
A\cup \phi   & = & A \\
A\cup \Omega & = &\Omega
\end{eqnarray*}
\end{itemize}
\end{property}

\begin{property}
Dado $A$ cualquier conjunto, entonces $\varnothing \subset A$.
\end{property}

\textbf{Demostración}: Sea $A$ cualquier conjunto. Por otra parte como el conjunto $\varnothing$ no contiene puntos, entonces es lógicamente correcto decir que todo punto que pertenece a $\varnothing$ también pertenece a $A$.



Note que en $\Omega$ evento, subconjunto o conjunto denotan lo mismo.

Observaciones:

\begin{remark}
Dados $A$ y $B$ eventos definidos en $\Omega$, el evento cuyos elementos ocurren en $A$ o en $B$, pero no en ambos está dado por 
\begin{equation*}
E=(A\cap B^{c})\cup (A^{c}\cap B)
\end{equation*}    
\end{remark}


\begin{remark}
El evento cuyos elementos son todos los elementos de $\Omega$ excepto aquellos que pertenecen a $A$ y a $B$, es decir, el evento cuyos elementos ocurrieron a lo más una vez en $\Omega$, está dado por
\begin{equation*}
F=(A\cap B)^{c}=A^{c}\cup B^{c}
\end{equation*}
\end{remark}


%%%%%%%%%%%%%%%%%%%%%%%%%%%%%%%%%%%%%%%%%%%%%%%%%%%%%%%%%%%%%%
\section{Definición de espacio muestral y álgebra de conjuntos}
 
%%%%%%%%%%%%%%%%%%%%%%%%%%%%%%%%
\subsection{Espacio muestral}

\begin{definition}
Un experimento es cualquier proceso real o hipotético en el cual los posibles resultados pueden ser identificados con anticipación por el investigador.
\end{definition}

Así un experimento es cualquier proceso que:
\begin{enumerate}
\item Puede ser repetido, al menos teóricamente, un número infinito o finito
de veces.
\item El conjunto de todos los posibles resultados del experimento están bien definidos. Es decir, todos sus posibles resultados pueden ser en listados, caracterizados o definidos de alguna forma.
\end{enumerate}

Cada uno de los posibles resultados del experimento se le llama resultado muestral y se denota con $\omega$; la totalidad de estos resultados se le llama \textbf{espacio muestral}. Es decir, los resultados del experimento componen el espacio muestral.

Así, un espacio muestral de un experimento puede ser pensado como un conjunto, colección o diferentes resultados posibles; y cada resultado puede ser pensado como un elemento del espacio muestral.

Ejemplos de experimentos:

\begin{example}
Lanzar una moneda una vez. \\
Resultados: águila, sol. \\ 
Espacio muestral: $\Omega=\{aguila, sol\}$.\\   
\end{example}

\begin{example}
Lanzar un dado una vez. \\
Resultados: $1,2,3,4,5,6$. \\
Espacio muestral: $\Omega=\{1,2,3,4,5,6\}$.\\   
\end{example}

\begin{example}
Lanzar una moneda dos veces. \\
Resultados: AA, AS, SA, SS.\\ 
Espacio muestral: $\Omega=\{AA,AS,SA,SS\}$.\\    
\end{example} 

\begin{example}
Número de nacimientos al año.\\
Los resultados se encontrarán en un intervalo.   
\end{example}
\begin{example}
Lanzar una moneda tres veces.\\
Resultados: SSS, ASS,SAS, AAS, SSA, ASA, SAA, AAA. \\
Espacio muestral: $\Omega=\{ SSS, ASS, SAS, AAS, SSA, ASA, SAA, AAA\}$.\\
\end{example}

\begin{example}
 Lanzar un dado dos veces.\\
Resultados: 
$\{1,1\}, \{2,1\}, \{3,1\},\{4,1\},\{5,1\},\{6,1\},\{1,2\},\{2,2\}\{3,2\},\{4,2\},$
$\{5,2\},\{6,2\},\{1,3\},\{2,3\},\{3,3\},\{4,3\},\{5,3\},\{6,3\},\{1,4\},\{2,4\},
\{3,4\},\{4,4\},$
$\{5,4\},\{6,4\},\{1,5\},\{2,5\},\{3,5\},\{4,5\},\{5,5\},\{6,5\},\{1,6\},\{2,6\},\{3,6\},\{4,6\},$
$\{5,6\},\{6,6\}$   
\end{example}


\begin{definition}
El \textbf{espacio muestral} denotado por $\Omega$, es la colección de todos los posibles resultados de un experimento
real o hipotético.
\end{definition}

Observaciones:
\begin{itemize}
\item  El espacio muestral de un experimento aleatorio no es único, ya que su constitución depende del investigador.
\item  El conjunto de resultados del experimento definen el espacio muestral $\Omega$; es decir, los elementos de $\Omega$ son los resultados del experimento.
\item  El tamaño del espacio muestral, también llamada \textbf{cardinalidad} de $\Omega$ está determinado por todos los  \textbf{posibles resultados del experimento}.
\end{itemize}

Dado $\Omega$ como un conjunto se pueden obtener de éste subconjuntos, los cuales pueden tener cero elementos, un elemento, dos elementos,...., todos los elementos de $\Omega$.

\begin{definition}
Cualquier subconjunto de $\Omega$  constituye un  \textbf{evento}.
\end{definition}

Así dado $\Omega =\{\omega_{1},\omega_{2,}\omega _{3},\omega_{4}\}$; algunos eventos de $\Omega$ son $A_1=\{\omega_{1}\}$, $A_2=\{\omega_{1},\omega_{3}\}$, $A_3=\{\omega_{1,}\omega_{3},\omega_{4}\}$ y $A_4=\Omega$

Así un evento de $\Omega$, es cualquier colección de los resultados posibles del experimento.

Los eventos con un elemento se llaman eventos simples, y a los eventos con más de un elemento se llaman eventos compuestos.
 \vspace{0.5cm}

El número total de subconjuntos (eventos) de $\Omega$ está dado por $2^{\text{número de elementos de }\Omega }=2^{\#\Omega}$.

Es decir, $2^{\#\Omega}$ nos indica el total de subconjuntos de $\Omega$.

\begin{example}
Si $\Omega =\{a,b,c,\}$, entonces $2^{3}=8$. Luego el número total de subconjuntos (eventos) de $\Omega$ es $8$; los cuales son $\{\Omega$, $\phi$, $\{a\},$ $\{b\},$ $\{c\},$ $\{a,b\},$ $\{a,c\},$ $\{b,c\}\}$.  
\end{example}


\begin{example}
Experimento lanzar una moneda una vez. \\    
Resultado del experimento: águila y Sol. \\
Espacio muestral: $\Omega=\{\acute{a}guila, Sol\}$. \\
El número total de eventos de $\Omega$ es $2^{2}=4$. \\
Eventos de $\Omega$: $\{\acute{a}guila\}$, $\{Sol\}$, $\Omega$, $\varnothing$.
\end{example}

Note que los eventos de $\Omega$, finalmente son conjuntos.

\begin{definition}
La clase de todos los eventos asociados con un experimento dado, es definido como el \textbf{espacio de eventos}. A este espacio de eventos se le denota con $S$.
\end{definition}

Cualquier subconjunto de $\Omega$ puede constituir el espacio de eventos, $S$, si cumple con las siguientes condiciones:

\begin{enumerate}
\item $\Omega \in S$.
\item Si $A\in S$ entonces $A^{c}\in S$.
\item Si $A_{1}$ y $A_{2}$ $\in S$ entonces $A_{1}\cup A_{2}\in S$.
\end{enumerate}

Cualquier colección de eventos con las tres propiedades anteriores, se le  conoce como  \textbf{álgebra de Boolean} o  \textbf{álgebra de eventos}.

%%%%%%%%%%%%%%%%%%%%%%%%%%%%%%%%%%%%%%%%%%%%%%%%%%%
\subsection{Práctica con r}

Instalación de paquetes.

\begin{lstlisting}
        install.packages("prob_1.0-1.tar.gz",
                 dependencies = TRUE,
                 repos = NULL)


       install.packages("fAsianOptions_3042.82.tar.gz",
                 dependencies = TRUE,
                 repos = NULL)


        install.packages("C:fOptions_3042.86.tar.gz",
                 dependencies = TRUE,
                 repos = NULL)

        library(fAsianOptions)
        library(fOptions)
        library(fBasics)
        library(timeSeries)
        library(combinat)
        library(prob)
\end{lstlisting}


Generación de espacio muestral

\begin{lstlisting}
        #---Lanzar una moneda 1 veces
       tosscoin(1)
\end{lstlisting}


\begin{lstlisting}
      #--- Lanzar un dado 1 veces
      rolldie(1)
\end{lstlisting}

\begin{lstlisting}
        #---Lanzar una moneda 2 veces
       tosscoin(2)
\end{lstlisting}


\begin{lstlisting}
        #---Lanzar una moneda 3 veces
       tosscoin(3)
\end{lstlisting}

\begin{lstlisting}
      #--- Lanzar un dado dos veces
      rolldie(2)
\end{lstlisting}

%%%%%%%%%%%%%%%%%%%%%%%%%%%%%%%%%%%%%%%%%%%%%%%%%
\section{Probabilidad de un evento}

¿Qué es la probabilidad de un evento?

\textbf{Interpretación subjetiva}: Es la probabilidad que asigna una persona a un posible resultado de algún proceso, dicha probabilidad representa su propio juicio. Este juicio está basado en lo que cree la persona en función de la información que tiene del proceso.

\vspace{5mm}

\textbf{Interpretación clásica}: Esta interpretación está basada en el concepto de resultados igualmente probables; si el resultado de algún proceso tiene que ser uno de los $n$ resultados diferentes y si estos $n$ resultados son igualmente probables de ocurrir, entonces la probabilidad de cada uno de los resultados es 
\begin{equation*}
\frac{1}{n}.
\end{equation*}

Una definición más formal de esta probabilidad está dada por lo siguiente: Si un experimento aleatorio puede resultar en $n$ resultados mutuamente excluyentes e igualmente probables, y si $n_{A}$ de esos resultados tienen el atributo $A$, entonces la probabilidad de $A$ está dado por 

\begin{equation*}
\frac{n_{A}}{n}
\end{equation*}


Ejemplo: el lanzamiento de un dado.

\vspace{5mm}
\textbf{Interpretación frecuentista}: Es la frecuencia relativa con la que se obtiene un resultado si el proceso se repite una gran cantidad de veces en condiciones similares. En otras palabras, si después de realizar $n$ veces el experimento, donde $n$ es grande, se observa que un evento ocurre en $k$ de estas veces; entonces la probabilidad del evento está dado por 

\begin{equation*}
\frac{k}{n}.
\end{equation*}

%%%%%%%%%%%%%%%%%%%%%%%%%%%%%%%%%%%%%%%%%%%%%%%%%
\subsection{Práctica en R} 


Lanzamiento de una moneda 100 veces.
\begin{lstlisting}
      # 100 tosses
       toss100 <- toss(coin1,
                       times = 100)

      # summary
      summary(toss100)
\end{lstlisting}
A esta frecuencia relativa se le llama probabilidad empírica del evento.


%%%%%%%%%%%%%%%%%%%%%%%%%%%%%%%%%%%%%%%%%%%%%%%%%
\subsection{Axiomas de probabilidad (Kolmogórov)} 

Sea $A$ cualquier evento definido en $\Omega$, el símbolo $P(A)$ denota la probabilidad de $A$. $P$ es una función y $P(A)$ es un número.

Si $\Omega$ tiene un número finito de resultados, entonces tres axiomas son necesarios y suficientes para caracterizar a $P$, la función de probabilidad.

Axioma 1:
Dado $A$ cualquier evento definido en $\Omega$, entonces 
\begin{equation*}
P(A)\geq 0.
\end{equation*}

Axioma 2:
$P(\Omega)=1$

Axioma 3:
Dado $A$ y $B$ dos eventos mutuamente excluyentes definidos en $\Omega$, entonces 
\begin{equation*}
P(A\cup B)=P(A)+P(B).
\end{equation*}


Si el número de elementos de $\Omega$ es infinito, entonces para $A_{1},A_{2},...,$ eventos definidos en $\Omega$ con $A_{i}\cap A_{j}=\varnothing$ para $i\neq j$, se tiene
\begin{equation*}
P\left( \bigcup\limits_{i=1}^{\infty }A_{i}\right) =\sum_{i=1}^{\infty}P(A_{i}).
\end{equation*}

\begin{definition}[Función de probabilidad]
Una función de probabilidad $P(\cdot )$ es una función con dominio $S$ (álgebra de eventos) y contradominio el intervalo $[0,1]$, la cual satisface los siguientes axiomas.
\begin{itemize}
    \item[A1] $P(A)\geq 0$, $\forall$ $A\in S$.
    \item[A2] $P(\Omega)=1$.
    \item[A3] Si $A_{1},A_{2},...,$ es una secuencia de eventos disjuntos en $S$ y si $\bigcup\limits_{i=1}^{\infty}A_{i}\in S$ entonces
\begin{equation*}
P\left( \bigcup\limits_{i=1}^{\infty }A_{i}\right) =\sum_{i=1}^{\infty}P(A_{i}).
\end{equation*}
\end{itemize}
\end{definition}

Sea $\Omega$ un espacio muestral con $n$ elementos. Es decir, 
\begin{equation*}
\Omega =\{\omega_{1},...,\omega_{n}\}
\end{equation*}

y sea $A_{i}$ un evento simple, es decir $\exists!$ $\omega _{i}\in A_{i}$, $i=1,...,n$; tal que para $i\neq j$  $A_{i}\cap A_{j}=\varnothing$.

Entonces:
\begin{itemize}
    \item[a)] $P(A_{i})\geq 0$ $\left/ \left/ Axioma\text{ }1\right. \right.$
    \item[b)] $\sum_{i=1}^{n}P(A_{i})=P(\Omega)$
\end{itemize}

Note que de b) se tiene que 
\begin{equation*}
\sum_{i=1}^{n}P(A_{i})=1\text{ }\left/ \left/ \text{Axioma }2\right.
\right.
\end{equation*}

entonces 
\begin{equation*}
0\leq P(A_{i})<1\text{, }i=1,...,
\end{equation*}

Si todos los $A_{i}$ tienen la misma probabilidad de ocurrir, entonces 
\begin{equation*}
P(A_{i})=\frac{1}{n}\text{, }i=1,...,n
\end{equation*}
y para cualquier evento compuesto, $B$, con $k$ elementos se tiene que la probabilidad de $B$, está dada por
\begin{equation*}
P(B)=\frac{k}{n}.
\end{equation*}

\begin{example}
 \textbf{Experimento}: Lanzar un dado una vez, definir tres eventos y calcular su probabilidad.

Resultados del experimento: $\Omega =\{1,\text{ }2,\text{ }3,\text{ }4,\text{ }5,\text{ }6\}$

Eventos y probabilidades:
\begin{eqnarray*}
A_{1} &=&\{1,2\};\text{ }P(A_{1})=2/6 \\
A_{2} &=&\{2,4,6\};\text{ }P(A_{2})=3/6 \\
A_{3} &=&\{3\};\text{ }P(A_{3})=1/6
\end{eqnarray*}   
\end{example}



\begin{definition}[Espacio de probabilidad]
Un espacio de probabilidad está dado por 
\begin{equation*}
(\Omega,S,P(\cdot ))
\end{equation*}
donde $\Omega$ es espacio muestral, $S$ es una colección de eventos (un álgebra) y $P(\cdot)$ es una función de probabilidad con dominio $S$ y contradominio $[0,1]$.
\end{definition}


%%%%%%%%%%%%%%%%%%%%%%%
\subsection{Resultados} 

\begin{property}
$P(\varnothing)=0$ (Larsen 2012, pag. 28).
\end{property}

\begin{property} Dados $A$ y $B$ conjuntos definidos en $\Omega $, entonces $P(A\cap B)=P(B\cap A)$.
\end{property}

\begin{property} Dados $A$ y $B$ conjuntos definidos en $\Omega $ tal que $A\cap B=\varnothing$, entonces $P(A\cap B)=0$.
\end{property}
Para las siguientes demostraciones, se asume que $\Omega$ y $S$ (álgebra de eventos) están dados y que $P(\cdot)$ es una función de probabilidad con dominio $S$ y contradominio $[0,1]$.

\begin{theorem}
Dado $A$ y $B$, dos eventos definidos en $\Omega$, entonces 
\begin{equation*}
P(A\cup B)=P(A)+P(B)-P(A\cap B)
\end{equation*}
\end{theorem}

\textbf{Demostración}: Note que 
\begin{equation*}
A=(A\cap B)\cup (A\cap B^{c})
\end{equation*}
\begin{equation*}
B=(B\cap A)\cup (B\cap A^{c})
\end{equation*}
entonces
\begin{eqnarray*}
P(A) &=&P(A\cap B)+P(A\cap B^{c})\text{ //Axioma 3} \\
P(B) &=&P(B\cap A)+P(B\cap A^{c})\text{ //Axioma 3}
\end{eqnarray*}
sumando
\begin{equation*}
P(A)+P(B)=P(A\cap B)+P(A\cap B^{c})+P(B\cap A)+P(B\cap A^{c})
\end{equation*}
despejando
\begin{equation*}
P(A)+P(B)-P(B\cap A)=P(A\cap B)+P(A\cap B^{c})+P(B\cap A^{c})
\end{equation*}
simplificando
\begin{equation*}
P(A)+P(B)-P(B\cap A)=P(A\cup B)
\end{equation*}%
Por lo tanto 
\begin{equation*}
P(A\cup B)=P(A)+P(B)-P(B\cap A)
\end{equation*}

\begin{theorem}
Sea $A$ un evento definido en $\Omega$, entonces 
\begin{equation*}
P(A^{c})=1-P(A).
\end{equation*}
\end{theorem}

\textbf{Demostración}: Note que 
\begin{equation*}
A\cup A^{c}=\Omega \text{ // Resultado}
\end{equation*}
entonces
\begin{equation}
P(A\cup A^{c})=P(\Omega )=1\text{ //Axioma 2}  \tag{ Obs1}
\end{equation}
como 
\begin{equation*}
A\cap A^{c}=\varnothing \text{ // Resultado}
\end{equation*}
entonces
\begin{eqnarray*}
P(A\cup A^{c}) &=&P(A)+P(A^{c})-P(A\cap A^{c})\text{ //Teorema} \\
&=&P(A)+P(A^{c})-P(\varnothing ) \\
&=&P(A)+P(A^{c})-0 \\
&=&P(A)+P(A^{c})=1\text{ //Obs1}
\end{eqnarray*}
Por lo tanto, 
\begin{equation*}
P(A^{c})=1-P(A)
\end{equation*}

\begin{theorem}
Sean $A$ y $B$ dos eventos definidos en $\Omega$ donde $A\subseteq B$ entonces $P(A)\leq P(B)$.
\end{theorem}

\textbf{Demostración}: Como 
\begin{equation*}
A\subseteq B
\end{equation*}
entonces
\begin{equation*}
B=A\cup (B\cap A^{c})
\end{equation*}
entonces
\begin{equation*}
P(B)=P(A)+P(B\cap A^{c})\text{ // Axioma 3 }A\cap (B\cap A^{c})=\varnothing 
\end{equation*}
como
\begin{equation*}
P(B\cap A^{c})\geq 0\text{, //Axioma 1}
\end{equation*}
entonces
\begin{equation*}
P(B)\geq P(A)
\end{equation*}
luego
\begin{equation*}
P(A)\leq P(B)
\end{equation*}

\begin{theorem}
Dado $A$ un evento definido en $\Omega$, entonces
\begin{equation*}
P(A)\leq 1
\end{equation*}
\end{theorem}

\textbf{Demostración}: Note que 
\begin{equation*}
A\subseteq \Omega
\end{equation*}
entonces
\begin{equation*}
P(A)\leq P(\Omega )=1
\end{equation*}
luego
\begin{equation*}
P(A)\leq 1
\end{equation*}


\begin{exercise}
Un estudiante es seleccionado de una clase, éste puede ser hombre o mujer. Si la probabilidad de que un chico sea seleccionado es $0.3$, ¿Cuál es la probabilidad de que una chica sea seleccionada?    
\end{exercise}

\begin{solution}
H: hombre\\
M: Mujer\\
Note\\
\begin{equation*}
M\cap H=\varnothing
\end{equation*}

\begin{equation*}
P(\Omega )=P(M\cup H)=P(M)+P(H)
\end{equation*}
entonces 
\begin{equation*}
1=0.3+P(M)
\end{equation*}
Por lo tanto
\begin{equation*}
P(M)=0.7
\end{equation*}
\end{solution}


\begin{exercise}
Si la probabilidad de que el estudiante $A$ repruebe un examen es $0.5$, la probabilidad de que un estudiante $B$ repruebe el examen es $0.2$, y la probabilidad de que ambos estudiantes reprueben el examen es $0.1$, ¿Cuál es la probabilidad de que al menos uno de los dos estudiantes repruebe el examen? 
\end{exercise}

\begin{solution}
 Identificación de eventos:
$P(A)=0.5$, $P(B)=0.2$, $P(A\cap B)=0.1$
\begin{eqnarray*}
Luego
P(A\cup B) &=&P(A)+P(B)-P(A\cap B) \\
           &=&0.5+0.2-0.1 \\
           &=&0.6
\end{eqnarray*}   
\end{solution}


%%%%%%%%%%%%%%%%%%%%%%%%%%%%%%%%%%%%%%%%%%%%%%%%%%
\section{Probabilidad condicional e independencia}

\begin{definition}[Probabilidad condicional]
Dado $A$ y $B$ eventos en $S$, de un espacio de probabilidad dado ($\Omega,S,P(\cdot)$). La probabilidad condicional del evento $A$ dado el evento $B$, está dado por 
\begin{equation*}
P(A\left\vert B\right. )=\frac{P(A\cap B)}{P(B)}\text{, }P(B)>0.
\end{equation*}
\end{definition}

Note que
\begin{equation*}
P(A\left\vert B\right. )=\frac{P(A\cap B)}{P(B)}
\end{equation*}
y 
\begin{equation*}
P(B\left\vert A\right. )=\frac{P(B\cap A)}{P(A)}
\end{equation*}
despejando ambas igualdades se tiene
\begin{equation*}
P(A\cap B)=P(B)P(A\left\vert B\right. )
\end{equation*}
\begin{equation*}
P(B\cap A)=P(A)P(B\left\vert A\right. )
\end{equation*}
como
\begin{equation*}
P(A\cap B)=P(B\cap A)
\end{equation*}
igualando se tiene
\begin{equation*}
P(B)P(A\left\vert B\right. )=P(A)P(B\left\vert A\right. )
\end{equation*}

\begin{remark}
$P(\cdot \left\vert B\right.)$ es una función de probabilidad con dominio $S$ y contradominio el intervalo $[0,1]$.    
\end{remark}

\textbf{Independencia de eventos}: $A$ es independiente de $B$ si $P(A\left\vert B\right.)=P(A)$.

\begin{definition}[Independencia de dos eventos]
Para un espacio de probabilidad dado ($\Omega ,S,P(\cdot)$), $A$ y $B$, eventos de $S$, son independientes si y solo si cualquiera de las siguientes condiciones se cumple:
\begin{enumerate}
\item $P(A\cap B)=P(A)P(B)$
\item  $P(A\left\vert B\right. )=P(A)$,  $P(B)>0$
\item  $P(B\left\vert A\right. )=P(B)$,  $P(A)>0$.
\end{enumerate}
\end{definition}

// Estadísticamente independientes

\begin{theorem}
Si $A$ y $B$ son dos eventos independientes, entonces $A$ y $B^{c}$ son eventos independientes.
\end{theorem}

\textbf{Demostración}: Note que para cualesquiera dos eventos $A$ y $B$,
\begin{equation*}
A=(A\cap B)\cup (A\cap B^{c})
\end{equation*}
Como 
\begin{equation*}
(A\cap B)\cap (A\cap B^{c})=\varnothing
\end{equation*}
entonces
\begin{equation*}
P(A)=P(A\cap B)+P(A\cap B^{c})\text{ //Axioma 3}
\end{equation*}

Despejando

\begin{eqnarray*}
P(A\cap B^{c}) &=&P(A)-P(A\cap B) \\
&=&P(A)[1-P(B)]\text{ //}A\text{ y }B\text{ independientes} \\
&=&P(A)P(B^{c})
\end{eqnarray*}
luego $A$ y $B^{c}$ son independientes.\\

Análogamente se puede probar que:
\begin{enumerate}
    \item $A^{c}$ y $B$ son independientes
    \item $A^{c}$ y $B^{c}$ son independientes
\end{enumerate}

\begin{definition}[Independencia de 3 eventos]
 Los eventos $A_{1},A_{2},A_{3}$ son mutuamente independientes si:
\begin{enumerate}
    \item $P(A_{j}\cap A_{k})=P(A_{j})P(A_{k})$, $j\neq k$, $j,k=1,2,3$
    \item$P(A_{1}\cap A_{2}\cap A_{3})=P(A_{1})P(A_{2})P(A_{3}).$
\end{enumerate}
\end{definition}

\begin{definition}[Independencia de eventos]
 Sea ($\Omega,S,P(\cdot )$) un espacio de probabilidad con $A_{1},A_{2},...,A_{n}$ $n$ eventos en $S$, estos son independientes si y solo sí 
\begin{eqnarray*}
P(A_{i}\cap A_{j}) &=&P(A_{i})P(A_{j})\text{, }i\neq j \\
P(A_{i}\cap A_{j}\cap A_{k}) &=&P(A_{i})P(A_{j})P(A_{k}),\text{ }i\neq
j,j\neq k,i\neq k \\
 & \vdots&\\
P\left( \bigcap\limits_{i=1}^{n}A_{i}\right)
&=&\prod\limits_{i=1}^{n}P(A_{i})
\end{eqnarray*}
\end{definition}

Sea $\Omega$ un espacio muestral, tal que 
\begin{equation*}
\bigcup\limits_{j=1}^{n}B_{j}=\Omega \text{, }B_{j}\cap
B_{i}=\varnothing \text{, }\forall \text{ }i\neq j
\end{equation*}

Ahora sea $A\in S$, plantear la probabilidad de $A$ en términos de las $B_{j}$.



